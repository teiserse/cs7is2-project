\documentclass{svproc}
\usepackage{url}
\def\UrlFont{\rmfamily}

\begin{document}
	\mainmatter
	\title{Analysis of different AI Strategies for Solving Picross Puzzles
	}
	\subtitle{CS7IS2 Project (2020/2021)}
	\author{Andrej Liadov, Edvinas Teiserskis, Min Wu, Edmond Cheng, Adam McQuade}
	
	\institute{
		\email{liadova@tcd.ie}, {teiserse@tcd.ie}, {wumi@tcd.ie}, {chenge@tcd.ie}, {amcquade@tcd.ie}}
	
	\maketitle              % typeset the title of the contribution
	
	\begin{abstract}
Picross, also known as Nonograms, is a wonderful series of games in which players use logic to solve puzzles that the cells in a grid must be marked or just left clear constrained by the numbers at the side of the grid. The least time to solve the puzzles is the key to win. In this case, our research provides a survey that analyzes and compares different search AI algorithms —— A-star, Q-learning, Depth-First Search(DFS), Constraint satisfaction problems (CSPs) to solve a 5x5 picross puzzles. Through the process of implementation and evaluation we prove the state-of-the-art of all these algorithms and we also find out their common aspects and differences, connections between methods, drawbacks and open problems. 
		\keywords{artificial intelligence, A-star, Q-learning, CSPs}
	\end{abstract}
	%
	
	

	\section{Introduction}
	In this section, you should introduce your work: what are the motivations behind this work? What is the relevant problem that you are investigating? Why is it relevant?
	Briefly, introduce the background information required to understand the problem and the concepts that you will develop.
	
	This section should also contain the link to the recording of your presentation (college OneDrive link – please make sure sharing permissions are such that everyone with tcd email can access it)
	
	\section{Related Work}
	In this section you will discuss possible approaches to solve the problem you are addressing, justifying your choice of the 3 you have selected to evaluate. Also, briefly introduce the approaches you are evaluating with a specific emphasis on differences and similarities to the proposed approach(es).
	
	\section{Problem Definition and Algorithm}
	This section formalises the problem you are addressing and the models used to solve it. This section should provide a technical discussion of the chosen/implemented algorithms. A pseudocode description of the algorithm(s) can also be beneficial to a clear explanation. It is also possible to provide one example that clarifies the way an algorithm works. It is important to highlight in this section the possible parameters involved in the model and their impact, as well as all the implementation choices that can impact the algorithm.
	
	\subsection{Subsection Title}
	
	\section{Experimental Results}
	This section should provide the details of the evaluation. Specifically:
	\begin{itemize}
		\item Methodology: describe the evaluation criteria, the data used during the evaluation, and the methodology followed to perform the evaluation.
		\item Results: present the results of the experimental evaluation. Graphical data and tables are two common ways to present the results. Also, a comparison with a baseline should be provided.
		\item Discussion: discuss the implication of the results of the proposed algorithms/models. What are the weakness/strengths of the method(s) compared with the other methods/baseline?
	\end{itemize}
	
	\section{Conclusions}
	Provide a final discussion of the main results and conclusions of the report. Comment on the lesson learnt and possible improvements.
	
	
	A standard and well formatted bibliography of papers cited in the report. For example:
	
	\begin{thebibliography}{6}
		%
		
		\bibitem {smit:wat}
		Smith, T.F., Waterman, M.S.: Identification of common molecular subsequences.
		J. Mol. Biol. 147, 195?197 (1981). \url{doi:10.1016/0022-2836(81)90087-5}
		
		\bibitem {may:ehr:stein}
		May, P., Ehrlich, H.-C., Steinke, T.: ZIB structure prediction pipeline:
		composing a complex biological workflow through web services.
		In: Nagel, W.E., Walter, W.V., Lehner, W. (eds.) Euro-Par 2006.
		LNCS, vol. 4128, pp. 1148?1158. Springer, Heidelberg (2006).
		\url{doi:10.1007/11823285_121}
		
		\bibitem {fost:kes}
		Foster, I., Kesselman, C.: The Grid: Blueprint for a New Computing Infrastructure.
		Morgan Kaufmann, San Francisco (1999)
		
		\bibitem {czaj:fitz}
		Czajkowski, K., Fitzgerald, S., Foster, I., Kesselman, C.: Grid information services
		for distributed resource sharing. In: 10th IEEE International Symposium
		on High Performance Distributed Computing, pp. 181?184. IEEE Press, New York (2001).
		\url{doi: 10.1109/HPDC.2001.945188}
		
		\bibitem {fo:kes:nic:tue}
		Foster, I., Kesselman, C., Nick, J., Tuecke, S.: The physiology of the grid: an open grid services architecture for distributed systems integration. Technical report, Global Grid
		Forum (2002)
		
		\bibitem {onlyurl}
		National Center for Biotechnology Information. \url{http://www.ncbi.nlm.nih.gov}
		
		
	\end{thebibliography}
\end{document}